\documentclass{article}
\usepackage{amsmath, fancyhdr, titling, array, tikz, booktabs, ulem}
\usepackage[plain]{algorithm}
\usepackage{algpseudocode}
\usepackage[top=1in,bottom=1in,left=1.5in,right=1.5in]{geometry}
\pagestyle{fancy}
\lhead{\theauthor}\chead{\thetitle}\rhead{\thedate} 

% ----- ENVIRONMENT ----- %
\newcounter{ProblemNum}
\newcounter{SubProblemNum}[ProblemNum]
\renewcommand{\theProblemNum}{\arabic{ProblemNum}}
\renewcommand{\theSubProblemNum}{\alph{SubProblemNum}}
\newcommand*{\anyproblem}[1]{\subsection*{#1}}
\newcommand*{\problem}[1]{\stepcounter{ProblemNum} %
\anyproblem{Problem \theProblemNum. \; #1}}
\newcommand*{\soln}[1]{\subsubsection*{#1}}
\newcommand*{\solution}{\soln{Solution}}
\renewcommand*{\part}{\stepcounter{SubProblemNum} %
\soln{Part (\theSubProblemNum)}}
\renewcommand{\theenumi}{(\alph{enumi})}
\renewcommand{\labelenumi}{\theenumi}
\renewcommand{\theenumii}{\roman{enumii}}
\newcommand{\alg}[1]{\textsc{\bfseries \footnotesize #1}}

% ----- INFO ----- %
\title{4.10 Undecidability Worksheet}
\author{Anghelo De La Cruz}
\date{\today}

% ----- DOCUMENT ----- %
\begin{document}

\subsection*{In your own words describe the characteristics that make
  a problem tractable or untractable.  }

A problem that is tractable is solvable in reasonable time. This means
that a solution can be found within polynomial time or below
$\mathcal{O}(2^n)$. A problem that is untractable is the opposite
meaning it cannot be solved within polynomial time and is
$\mathcal{O}(2^n)$ or above.

\subsection*{What is the difference between problems in the class “P”
  and problems in the class “NP”?}





\end{document}